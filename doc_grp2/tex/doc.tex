\documentclass[a4paper,12pt]{report}

\usepackage[utf8]{inputenc} % damit auch äöüß gehen
\usepackage[T1]{fontenc}	% so kann man bessere wörtliche rede machen
\usepackage[ngerman]{babel} % deutsche Silbentrennung
\usepackage{gensymb} % für besondere Symbole wie zB Grad -> \degree

\usepackage{csvsimple} % for tables

\usepackage{graphicx}

\usepackage{enumitem} % um aufzählungen zu machen
\usepackage[
	colorlinks,
	pdfpagelabels,
	bookmarksopen = true,
	bookmarksnumbered = true,
	linkcolor = black,
	plainpages = false,
	hypertexnames = false,
	citecolor = black
]{hyperref} % verlinkt referenzen für die digitale verwendung



%TODO: Einführung, Gruppenfoto (mit Auto), Passiver Schreibstil, Bilder, Allgemeine Beschreibungen der Kapitel, Autor der jeweiligen Funktion, Fazit, Zwischenfazit für jedes Kaptiel?, Sinnabschnitte,  Literatur/Quellverzeichnis



\begin{document}
	
%%%%%%%%%%%%%%%%%%%%% HEADER %%%%%%%%%%%%%%%%%%%%%

	\title{Robotik Praktikum -- Gruppe 2}
	\author{Frauke Jörgens (minf101207) \and Franz Wernicke (ite101729) \and Thorger Dittmann (ite101646) \and Jan Ottmüller (tinf101737) \and Felix Maaß (ite101754)}
	\date{\today}
	\maketitle
	
	\tableofcontents
	

%%%%%%%%%%%%%%%%%%%%%%%%%%%%%%%%%%%%%%%%%%%%%%%%%%	
\chapter{Allgemeine Themen}
\section{Lineare Funktion}

	Für manche Werte war es nötig eine Modulation hinzufügen zu können.
	
	Es wurde sich aus Gründen der besseren Visualisierung für einen eigenen ADTF Filter entschieden.
	Dieser nimmt ein Eingangssignal entgegen, welches daraufhin mit einem Gain und Offset versehen wird und anschließend den modulierten Wert zurückliefert.
	
	Diese Funktionalität entspricht einer "Linearen Funktion":
	
	\[y = g * x + o\]
	
	wobei $x$ der Eingang, $g$ der Gain, $o$ der Offset und $y$ der Ausgang ist.
	\\
	Gain und Offset können hierbei sowohl über eigene Eingangspins als auch über variable Filterparameter gesetzt werden.

\subsection{In Entwicklung}

	Es ist geplant, dass die Geschwindigkeit des Autos mithilfe des aktuellen Lenkwinkels in Kurven angepasst werden kann.

%%%%%%%%%%%%%%%%%%%%%%%%%%%%%%%%%%%%%%%%%%%%%%%%%%
\chapter{Lenkung}

%TODO: Hier ein bisschen was über die beiden Filter schreiben + Ackermann Steering Approximation + Theoretische Überlegungen auf zettel und papier
%https://en.wikipedia.org/wiki/Steering

%%%%%%%%%%%%%%%%%%%%%%%%%%%%%%%%%%%%%%%%%%%%%%%%%%
\chapter{Motorregelung}	

%%%%%%%%%%%%%%%%%%%%%%%%%%%%%%%%%%%%%%%%%%%%%%%%%%
\chapter{Kollisionsvermeidung}

	Mithilfe der Ultraschallsensoren sind vorn und hinten Hindernisse zu erkennen.
	Dadurch soll es möglich sein, zu reagieren bevor eine Kollision stattfindet.

\section{Genauigkeit der Ultraschallsensoren}

	\paragraph{Situation.}
	Die Ultraschallsensoren liefern einzeln betrachtet recht akkurate Werte.
	Die Genauigkeit wurde zu $\pm1cm$ bestimmt.
	
	\paragraph{Problem.}
	Leider ist es jedoch der Fall, dass die Messungen Ausreißer aufweisen.
	Teils wurden leicht erkennbare Fehler zurückgegeben ($-1$). Teils wichen die Werte aber auch um erhebliche Beträge ab.
	
	Den Grund für Letzteres diagnostizierten wir in der gleichzeitigen Ansteuerung der Sensoren.
	Da jedoch der Code der Arduinos (welche sich um die Erfassung der Werte kümmern) nicht offen liegt und für uns somit anpassbar wäre, können wir vorerst keine Änderungen daran vornehmen.
	
	Damit diese falschen Werte dennoch nicht mit in kommende Berechnungen einbezogen werden, musste eine entsprechende Lösung gefunden werden.

	\paragraph{Resultat.}
	Zur Stabilisierung der Messwerte entschieden wir uns für eine Filterung.
	Es wurde nach einem Filter gesucht, der robust gegenüber Ausreißern ist.
	Hier kam uns als Erstes der Median in den Sinn.
	
	Ein entsprechender Filter wurde nun sowohl als C++ Klasse als auch als ADTF Filter erstellt.

	Der ADTF Filter besitzt einen Parameter, der die Fenstergröße einstellt (i.e. die Anzahl an Werten über die der Median gebildet werden soll) und ermöglicht damit eine flexible Filterung der Messwerte.

\section{Hinderniserkennung}
	
	Mithilfe der aufbereiteten Ultraschallsensordaten wird es möglich sein, das Auto sicher durch die Umgebung zu navigieren.
	\footnote{Dieses Feature befindet sich noch in Entwicklung.}
	

%%%%%%%%%%%%%%%%%%%%%%%%%%%%%%%%%%%%%%%%%%%%%%%%%%
\chapter{Kollisionserkennung}

	Mithilfe der Daten des Accelerometers werden Kollisionen erkannt und ein sofortiger Nothalt ausgelöst.
	\footnote{Dieses Feature befindet sich noch in Entwicklung.}
	
	Dabei wird darauf geachtet, dass die Ausschläge der Werte entsprechend hoch und kurzweilig sind, da auch bei starkem Abbremsen bereits starke aber anhaltende Beschleunigungen entstehen.
	

%%%%%%%%%%%%%%%%%%%%%%%%%%%%%%%%%%%%%%%%%%%%%%%%%%	
\chapter{Fahrbahnerkennung}

\section{Linienerkennung (Binarisierung)}
	Threshold, HSV-Modell, Closing, ...

\section{Canny Edge Detection} %TODO vorher/nachher Bilder

\section{Hough Line Transformation}

\section{Line-Clustering}

\section{Klassifizierung}

\subsection{Haltelinien}


	
\end{document}


