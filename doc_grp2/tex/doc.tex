\documentclass[a4paper,12pt]{report}

\usepackage[utf8]{inputenc} % damit auch äöüß gehen
\usepackage[T1]{fontenc}	% so kann man bessere wörtliche rede machen
\usepackage[ngerman]{babel} % deutsche Silbentrennung
\usepackage{gensymb} % für besondere Symbole wie zB Grad -> \degree

\usepackage{csvsimple} % for tables

\usepackage{graphicx}

\usepackage{enumitem} % um aufzählungen zu machen
\usepackage[
	colorlinks,
	pdfpagelabels,
	bookmarksopen = true,
	bookmarksnumbered = true,
	linkcolor = black,
	plainpages = false,
	hypertexnames = false,
	citecolor = black
]{hyperref} % verlinkt referenzen für die digitale verwendung



%TODO: Einführung, Gruppenfoto (mit Auto), Passiver Schreibstil, Bilder, Allgemeine Beschreibungen der Kapitel, Autor der jeweiligen Funktion, Fazit, Zwischenfazit für jedes Kaptiel?, Sinnabschnitte,  Literatur/Quellverzeichnis



\begin{document}
	\title{Robotik Praktikum -- Gruppe 2}
	\author{Frauke Jörgens (minf101207) \and Franz Wernicke (ite101729) \and Thorger Dittmann (ite101646) \and Jan Ottmüller (tinf101737) \and Felix Maaß (ite101754)}
	\date{\today}
	\maketitle
	
	\tableofcontents
	
	
	
	
	\chapter{Allgemeine Filter}
	\section{Medianfilter}
	
		Zur Stabilisierung von Messwerten haben wir uns dafür entschieden den Medien Filter zu benutzen weil dieser ausreißersicher ist und damit falsche Werte wie zum Beispiel die -1, die die Ultraschallsensoren zeitweise ausgeben, nicht mit in kommende Berechnung einbezogen werden.
		
		So einen Filter habe ich nun sowohl als Klasse als auch als ADTF Filter erstellt und damit ist uns eine flexible Filterung möglich.
	
	
	\section{Lineare Funktion}
	
		Zur besseren Visualisierung habe ich einen ADTF Filter erstellt, der float Werte mit einem Gain und einem Offset versehen kann. 
		Gain und Offset können sowohl als Parameter im Filter selbst gesetzt als auch über Inputs Pins bestimmt werden.
	
	\chapter{Hinderniserkennung}
	
		Mithilfe der Ultraschall Sensoren werden vorne und hinten Hindernisse er kann ich die eventuell dem Auto im Weg stehen könnten.
		Dadurch ist es möglich, rechtzeitig zu reagieren bevor eine Kollision stattfindet.
	
	\chapter{Kollisionserkennung}
	
		Mithilfe der Daten des Accelerometers können wir Kollisionen erkennen und einen sofortigen Nothalt auslösen.
		Dabei wird darauf geachtet, dass die Ausschläge der Werte entsprechend hoch und kurzweilig sind, da auch bei starkem Abbremsen bereits starke aber anhaltende Beschleunigungen entstehen.
	
	\chapter{Fahrbahnerkennung}
	
	\section{Linienerkennung (Binarisierung)}
	Threshold, HSV-Modell, Closing, ...
	
	\section{Canny Edge Detection} %TODO vorher/nachher Bilder
	
	\section{Hough Line Transformation}
	
	\section{Line-Clustering}
	
	\section{Klassifizierung}
	
	\subsection{Haltelinien}
	
\end{document}


